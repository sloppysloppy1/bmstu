\lstset{ %
	language=prolog,                 % выбор языка для подсветки (здесь это С++)
	basicstyle=\small\sffamily, % размер и начертание шрифта для подсветки кода
	numbers=left,               % где поставить нумерацию строк (слева\справа)
	numberstyle=\tiny,           % размер шрифта для номеров строк
	stepnumber=1,                   % размер шага между двумя номерами строк
	numbersep=5pt,                % как далеко отстоят номера строк от подсвечиваемого кода
	showspaces=false,            % показывать или нет пробелы специальными отступами
	showstringspaces=false,      % показывать или нет пробелы в строках
	showtabs=false,             % показывать или нет табуляцию в строках
	frame=single,              % рисовать рамку вокруг кода
	tabsize=2,                 % размер табуляции по умолчанию равен 2 пробелам
	captionpos=t,              % позиция заголовка вверху [t] или внизу [b] 
	breaklines=true,           % автоматически переносить строки (да\нет)
	breakatwhitespace=false, % переносить строки только если есть пробел
	escapeinside={\#*}{*)}   % если нужно добавить комментарии в коде
}
\newpage

\newpage
\section*{Задание 1}
\addcontentsline{toc}{section}{\tocsecindent{Задание 1}}

\Large Составить программу – базу знаний, с помощью которой можно определить, например, множество студентов, обучающихся в одном ВУЗе. Студент может одновременно обучаться в нескольких ВУЗах. Привести примеры возможных вариантов вопросов и варианты ответов (не менее 3-х). Описать порядок формирования вариантов ответа. Исходную базу знаний сформировать с помощью только фактов.  
\newline

\lstinputlisting[
language = Prolog,
caption  = {Задание 1},
]{src/1.pro}

\section*{Задание 2}
\addcontentsline{toc}{section}{\tocsecindent{Задание 2}}

\Large Ту же базу знаний сформировать, используя правила.
  


\lstinputlisting[
language = Prolog,
caption  = {Задание 2},
]{src/2.pro}

\section*{Задание 3}
\addcontentsline{toc}{section}{\tocsecindent{Задание 3}}

\Large Разработать свою базу знаний (содержание произвольно). 
\\Данная база содержит описание машин.



\lstinputlisting[
language = Prolog,
caption  = {Задание 3},
]{src/3.pro}

    