% ответы на вопросы
\newpage
\section*{Ответы на вопросы}
\addcontentsline{toc}{section}{\tocsecindent{Ответы на вопросы}}
\begin{enumerate}
    \item \textbf{Что такое терм?}\\
    Терм - основная конcтрукция Prolog. Термы бывают простыми (константы (атомы и числа) и переменные) и составными. Обрабатываются слева направо.
    
    \item \textbf{Что такое предикат в матлогике (математике)?}\\
    Определенным на множествах $M_1,M_2,\ldots,M_n$ $n$-местным предикатом называется предложение, содержащее $n$ переменных $x_1,x_2,\ldots,x_n$, превращающееся в высказывание при подстановке вместо этих переменных любых конкретных элементов из множеств $M_1,M_2,\ldots,M_n$ соответственно.
    
    \item \textbf{Что описывает предикат в Prolog?}\\
    В Prolog предикат - это утверждение, истинность которого зависит от значения переменных, входящих в него. Синтаксис:
    \begin{enumerate}
    	\item :- означает if;
    	\item $,$ означает and;
    	\item $;$ означает or;
    	\item $.$ означает end.
	\end{enumerate}

    \item \textbf{Назовите виды предложений в программе и приведите примеры таких предложений из Вашей программы. Какие предложения являются основными, а какие – не основными?  Каковы: синтаксис и семантика (формальный смысл) этих предложений (основных и неосновных)?}\\
    Виды предложений:
    \begin{enumerate}
    	\item Факт - предложение, которое устанавливает безусловно-истинное отношение между термами, или утверждает некоторую безусловную истину. Пример:
\begin{lstlisting}[style = Prolog-pygsty] 
person("klyuge", "+7912571026", ad("volgograd", "varshavshkoe sh.", 52, 83)).
\end{lstlisting}
    	\item Правило - предложение, которое устанавливает отношение между термами при выполнении заданных условий. Пример:
\begin{lstlisting}[style = Prolog-pygsty] 
get_cars_by_number(Ph_num, Car_brand, Car_cost) :- person(L_name, Ph_num,
    		ad(Town, _, _, _)), (L_name, Car_brand, _, Car_cost, _, Town).
\end{lstlisting}
    	\item Запрос - предложение, которое устанавливает совпадения с фактами или правилами БЗ. Пример:
\begin{lstlisting}[style = Prolog-pygsty] 
get_cars_by_number(Ph_number, Car_brand, _), Ph_number = "+79154045900", nl.
\end{lstlisting}
    \end{enumerate}
	Факты являются основными предложениями в Prolog.
	
    \item \textbf{Каковы назначение, виды и особенности использования переменных в программе на Prolog? Какое предложение БЗ сформулировано в более общей – абстрактной форме: содержащее или не содержащее переменных?}\\
   	Переменные начинаются с заглавной буквы или с символа подчёркивания и могут быть анонимными или именованными (анонимная переменная обозначается нижним подчеркивание). Говорят, что переменная может быть связана с некоторым значением или оставаться независимой. Именованные переменные уникальны в рамках одного предложения. Анонимная переменная уникальна всегда. Переменные предназначены для передачи значений «во времени и в пространстве». 
   	
   	Предложение БЗ сформулировано в более общей абстрактной форме, если оно содержит переменные.
    
    \item \textbf{Что такое подстановка?}\\
    Подстановка - множество пар вида {$x_i$ = $t_i$}, где
    \begin{enumerate}
    	\item $x_i$ - переменная;
    	\item $t_i$ - терм.
    \end{enumerate}
    
    Если существует $A(x_1, x_2, ..., x_n)$ и подстановка $\theta = {x_1 = t_1, x_2 = t_2, ... , x_n = t_n}$ то применение подстановки заключается в замене $x_i$ на $t_i$.
    
    
    
    \item \textbf{Что такое пример терма? Как и когда строится? Как Вы думаете, система строит и хранит примеры?}\\
    
    Терм B называется примером терма $A$, если существует такая подстановка такая, что $B = A  \theta$.
    
    Терм C является общим примером $A, B$ если существует подстановки $\theta_1$ и $\theta_2$ такие что С$ С = A \theta_1$ и $C = B \theta_2$.
    
    В процессе выполнения программы — система, используя встроенный алгоритм унификации, пытается обосновать возможность истинности вопроса, строя подстановки и примеры термов (вопроса и формулировки знания), используя базу знаний, и найти такие значения переменных, при которых это удается, а значит, на поставленный вопрос можно дать ответ «Да». Возможно система «ошибается» в своих обоснованиях и возникает тупиковая ситуация, или, ответив на вопрос, пытается найти другой способ доказательства. Тогда включается механизм отката (отказа от последнего заключения (какого?) и последних действий, сделанных системой) и выполняется ре- конкретизация переменных, конкретизация которых была выполнена на последнем шаге.
    

\end{enumerate}

% ответ
%\\