% --- ugly internals for language definition ---
%
\makeatletter
\lstset{
	captionpos = below,
	frame      = single,
	columns    = fullflexible,
	basicstyle = \small\sffamily,
}
% initialisation of user macros
\newcommand\PrologPredicateStyle{}
\newcommand\PrologVarStyle{}
\newcommand\PrologAnonymVarStyle{}
\newcommand\PrologAtomStyle{}
\newcommand\PrologOtherStyle{}
\newcommand\PrologCommentStyle{}

% useful switches (to keep track of context)
\newif\ifpredicate@prolog@
\newif\ifwithinparens@prolog@

% save definition of underscore for test
\lst@SaveOutputDef{`_}\underscore@prolog

% local variables
\newcount\currentchar@prolog

\newcommand\@testChar@prolog%
{%
	% if we're in processing mode...
	\ifnum\lst@mode=\lst@Pmode%
	\detectTypeAndHighlight@prolog%
	\else
	% ... or within parentheses
	\ifwithinparens@prolog@%
	\detectTypeAndHighlight@prolog%
	\fi
	\fi
	% Some housekeeping...
	\global\predicate@prolog@false%
}

% helper macros
\newcommand\detectTypeAndHighlight@prolog
{%
	% First, assume that we have an atom.
	\def\lst@thestyle{\PrologAtomStyle}%
	% Test whether we have a predicate and modify the style accordingly.
	\ifpredicate@prolog@%
	\def\lst@thestyle{\PrologPredicateStyle}%
	\else
	% Test whether we have a predicate and modify the style accordingly.
	\expandafter\splitfirstchar@prolog\expandafter{\the\lst@token}%
	% Check whether the identifier starts by an underscore.
	\expandafter\ifx\@testChar@prolog\underscore@prolog%
	% Check whether the identifier is '_' (anonymous variable)
	\ifnum\lst@length=1%
	\let\lst@thestyle\PrologAnonymVarStyle%
	\else
	\let\lst@thestyle\PrologVarStyle%
	\fi
	\else
	% Check whether the identifier starts by a capital letter.
	\currentchar@prolog=65
	\loop
	\expandafter\ifnum\expandafter`\@testChar@prolog=\currentchar@prolog%
	\let\lst@thestyle\PrologVarStyle%
	\let\iterate\relax
	\fi
	\advance \currentchar@prolog by 1
	\unless\ifnum\currentchar@prolog>90
	\repeat
	\fi
	\fi
}
\newcommand\splitfirstchar@prolog{}
\def\splitfirstchar@prolog#1{\@splitfirstchar@prolog#1\relax}
\newcommand\@splitfirstchar@prolog{}
\def\@splitfirstchar@prolog#1#2\relax{\def\@testChar@prolog{#1}}

% helper macro for () delimiters
\def\beginlstdelim#1#2%
{%
	\def\endlstdelim{\PrologOtherStyle #2\egroup}%
	{\PrologOtherStyle #1}%
	\global\predicate@prolog@false%
	\withinparens@prolog@true%
	\bgroup\aftergroup\endlstdelim%
}

% language name
\newcommand\lang@prolog{Prolog-pretty}
% ``normalised'' language name
\expandafter\lst@NormedDef\expandafter\normlang@prolog%
\expandafter{\lang@prolog}

% language definition
\expandafter\expandafter\expandafter\lstdefinelanguage\expandafter%
{\lang@prolog}
{%
	language            = Prolog,
	keywords            = {},      % reset all preset keywords
	showstringspaces    = false,
	alsoletter          = (,
	alsoother           = @$,
	moredelim           = **[is][\beginlstdelim{(}{)}]{(}{)},
	MoreSelectCharTable =
	\lst@DefSaveDef{`(}\opparen@prolog{\global\predicate@prolog@true\opparen@prolog},
}

% Hooking into listings to test each ``identifier''
\newcommand\@ddedToOutput@prolog\relax
\lst@AddToHook{Output}{\@ddedToOutput@prolog}

\lst@AddToHook{PreInit}
{%
	\ifx\lst@language\normlang@prolog%
	\let\@ddedToOutput@prolog\@testChar@prolog%
	\fi
}

\lst@AddToHook{DeInit}{\renewcommand\@ddedToOutput@prolog{}}

\makeatother
%
% --- end of ugly internals ---


% --- definition of a custom style similar to that of Pygments ---
% custom colors
\definecolor{PrologPredicate}{RGB}{000,031,255}
\definecolor{PrologVar}      {RGB}{024,021,125}
\definecolor{PrologAnonymVar}{RGB}{000,127,000}
\definecolor{PrologAtom}     {RGB}{186,032,032}
\definecolor{PrologComment}  {RGB}{063,128,127}
\definecolor{PrologOther}    {RGB}{000,000,000}

% redefinition of user macros for Prolog style
\renewcommand\PrologPredicateStyle{\color{PrologPredicate}}
\renewcommand\PrologVarStyle{\color{PrologVar}}
\renewcommand\PrologAnonymVarStyle{\color{PrologAnonymVar}}
\renewcommand\PrologAtomStyle{\color{PrologAtom}}
\renewcommand\PrologCommentStyle{\itshape\color{PrologComment}}
\renewcommand\PrologOtherStyle{\color{PrologOther}}

% custom style definition 
\lstdefinestyle{Prolog-pygsty}
{
	language     = Prolog-pretty,
	upquote      = true,
	stringstyle  = \PrologAtomStyle,
	commentstyle = \PrologCommentStyle,
	literate     =
	{:-}{{\PrologOtherStyle :-}}2
	{,}{{\PrologOtherStyle ,}}1
	{.}{{\PrologOtherStyle .}}1
}



\newpage
\section*{Задание}
\addcontentsline{toc}{section}{\tocsecindent{Задание}}

\Large Составить программу, т.е. модель предметной области – базу знаний, объединив в ней информацию – знания:
\begin{enumerate}

	\item «Телефонный справочник»: Фамилия, №тел, Адрес – структура (Город, Улица, №дома, №кв);
	\item «Автомобили»: Фамилия владельца, Марка, Цвет, Стоимость, и др;
	\item « Вкладчики банков»: Фамилия, Банк, счет, сумма, др.
\end{enumerate}
Владелец может иметь несколько телефонов, автомобилей, вкладов (Факты).
Используя правила, обеспечить возможность поиска:
\begin{enumerate}
\item	а) По № телефона найти: Фамилию, Марку автомобиля, Стоимость автомобиля (может быть несколько),
\\б) Используя сформированное в пункте а) правило, по № телефона найти: только Марку автомобиля (автомобилей может быть несколько),
\item	Используя простой, не составной вопрос: по Фамилии (уникальна в городе, но в разных городах есть однофамильцы) и Городу проживания найти:  Улицу проживания, Банки, в которых есть вклады и № телефона.
\end{enumerate}
Для задания 1 и задания 2: 
для одного из вариантов ответов, и для а) и для б), описать словесно порядок поиска ответа на вопрос, указав, как выбираются знания, и, при этом, для каждого этапа унификации, выписать подстановку – наибольший общий унификатор, и соответствующие примеры термов.

\newpage
\section*{Листинг программы}
\addcontentsline{toc}{section}{\tocsecindent{Листинг программы}}
Ниже представлен листинг программы:
\lstinputlisting[
style      = Prolog-pygsty,
caption    = {Задания 1 и 2},
]{src/1.pro}

\section*{Описание порядка поиска ответов}
\addcontentsline{toc}{section}{\tocsecindent{Описание порядка поиска ответов}}
\begin{table}
	\caption{Вывод Ph\_number=+79154045900, Car\_brand=mercedes, Car\_cost=33637427 }
\begin{tabular}{|p{1cm}|p{5cm}|p{5cm}|p{5cm}|}
	\hline
	№ шага & Сравниваемые термы; подстановка, если есть  & Результат  & Дальнейшие действия: прямой ход или откат (к чему приводит?) \\
	\hline
	1 & Подстановка Ph\_number = '+79154045900' в запрос get\_cars\_by\_number (Ph\_number, Car\_brand, Car\_cost) & get\_cars\_by\_number ('+79154045900', Car\_brand, Car\_cost) & прямой ход \\
	\hline
	2 & Поиск совпадений и подстановка '+79154045900', Car\_brand, Car\_cost в соответственное правило  get\_cars\_by\_number(Ph\_num, Car\_brand, Car\_cost) & get\_cars\_by\_number( '+79154045900', Car\_brand, Car\_cost) & прямой ход \\
	\hline
	3 & Подстановка Ph\_num = '+79154045900' в person(L\_name, Ph\_num, ad(Town, \_, \_, \_)) & Создается пример person(L\_name, '+79154045900', ad(Town, \_, \_, \_)) & прямой ход \\
	\hline
	
	4 & Поиск person по примерy & person('klyuge', '+7912571026', ad('volgograd', 'varshavshkoe sh.', 52, 83))., не подходит. & откат \\
	\hline
	5 & Поиск person по примерy & ... & откат \\
	\hline
	6 & Поиск person по примерy & person('mishkin', '+79154045900', ad('kiev', 'pushkina', 25, 278)), L\_name = 'mishkin', Town = 'kiev' & прямой ход \\
	\hline
	7 & Подстановка в car(L\_name, Car\_brand, \_, Car\_cost, \_, Town) L\_name = 'mishkin', Car\_brand = Car\_brand, Car\_cost, Town = 'kiev' & Создается пример car('mishkin', Car\_brand, \_, Car\_cost, \_, Town, 'kiev') & прямой ход \\
	\hline
	8 & Поиск car по примеру & car('zayceva', 'geely', 'white', 6409914, 2015, 'kasimov'), не подходит & откат \\
	\hline
	9 & Поиск car по примеру & ... & откат \\
	\hline
	10 & Поиск car по примеру & car('mishkin', 'mercedes', 'black', 33637427, 2015, 'kiev'), Car\_brand = 'mercedes', Car\_cost = 33637427 & прямой ход \\
	\hline
	11 & Общий пример найден, вывод & Ph\_num = '+79154045900', Car\_brand = 'mercedes', Car\_cost = 33637427 &  \\
	\hline
\end{tabular}
\end{table}

\begin{table}
	\caption{Вывод Ph\_number=+79154045900, Car\_brand=mercedes }
	\begin{tabular}{|p{1cm}|p{5cm}|p{5cm}|p{5cm}|}
		\hline
		№ шага & Сравниваемые термы; подстановка, если есть & Результат & Дальнейшие действия: прямой ход или откат (к чему приводит?) \\
		\hline
		1 & Подстановка Ph\_number = '+79154045900' в запрос get\_cars\_by\_number (Ph\_number, Car\_brand, \_) & get\_cars\_by\_number( '+79154045900', Car\_brand, \_) & прямой ход \\
		\hline
		2 & Поиск совпадений и подстановка '+79154045900', Car\_brand в соответственное правило  get\_cars\_by\_number(Ph\_num, Car\_brand, \_) & get\_cars\_by\_number( '+79154045900', Car\_brand, \_) & прямой ход \\
		\hline
		3 & Подстановка Ph\_num = '+79154045900' в person(L\_name, Ph\_num, ad(Town, \_, \_, \_)) & Создается пример person(L\_name, '+79154045900', ad(Town, \_, \_, \_)) & прямой ход \\
		\hline
		4 & Поиск person по примерy & person('klyuge', '+7912571026', ad('volgograd', 'varshavshkoe sh.', 52, 83)), не подходит. & откат \\
		\hline
		5 & Поиск person по примерy & ... & откат \\
		\hline
		6 & Поиск person по примерy & person('mishkin',  '+79154045900', ad('kiev', 'pushkina', 25, 278)), L\_name = 'mishkin', Town = 'kiev' & прямой ход \\
		\hline
		7 & Подстановка в car(L\_name, Car\_brand, \_, \_, \_, Town) L\_name = 'mishkin', Car\_brand = Car\_brand, Car\_cost, Town = 'kiev' & Создается пример car('mishkin', Car\_brand, \_, \_, \_, Town, 'kiev') & прямой ход \\
		\hline
		8 & Поиск car по примеру & car('zayceva', 'geely', 'white', 6409914, 2015, 'kasimov'), не подходит & откат \\
		\hline
		9 & Поиск car по примеру & ... & откат \\
		\hline
		10 & Поиск car по примеру & car('mishkin', 'mercedes', 'black', 33637427, 2015, 'kiev'), Car\_brand = 'mercedes' & прямой ход \\
		\hline
		11 & Общий пример найден, вывод & Ph\_num = '+79154045900', Car\_brand = "mercedes" &  \\
		\hline
	\end{tabular}
\end{table}
\begin{table}
	\caption{Вывод Lname=toporova, Town=saint-petersburg, Street=varshavshkoe sh., Bank=tinkoff, Tel\_num=+79165072034 }
	\begin{tabular}{|p{1cm}|p{5cm}|p{5cm}|p{5cm}|}
		\hline
		№ шага & Сравниваемые термы; подстановка, если есть & Результат & Дальнейшие действия: прямой ход или откат (к чему приводит?) \\
		\hline
		1 & Подстановка Lname = 'toporova' и Town = 'saint-petersburg' в запрос get\_stuff(Lname, Town, Street, Bank, Tel\_num) & get\_stuff('toporova', 'saint-petersburg', Street, Bank, Tel\_num) & прямой ход \\
		\hline
		2 & Поиск совпадений и подстановка Lname = 'toporova', Town = 'saint-petersburg' Street = Street, Bank = Bank, Tel\_num = Tel\_num в соответственное правило  get\_stuff(Lname, Town, Street, Bank, Tel\_num) & ('toporova', 'saint-petersburg', Street, Bank, Tel\_num) & прямой ход \\
		\hline
		3 & Подстановка Lname = 'toporova', Town = 'saint-petersburg' в person(L\_name, Tel\_num, ad(Town, Street, \_, \_)) & Создается пример person('toporova', Tel\_num, ad('saint-petersburg', Street, \_, \_)) & прямой ход \\
		\hline
		4 & Поиск person по примерy & person('klyuge', '+7912571026', ad('volgograd', 'varshavshkoe sh.', 52, 83)), не подходит & откат \\
		\hline
		5 & Поиск person по примерy & ... & откат \\
		\hline
		6 & Поиск person по примерy & person('toporova', '+79165072034', ad('saint-petersburg', 'varshavshkoe sh.', 100, 1151)), L\_name = 'toporova', Tel\_num = '+79165072034', Town = 'saint-petersburg', Street = 'varshavshkoe sh.' & прямой ход \\
		\hline
		7 & Подстановка в bank\_client(Lname, Bank, \_, \_, \_, Town) L\_name = 'toporova', Bank = Bank, Town = 'saint-petersburg' & Создается пример bank\_client('toporova', Bank, \_, \_, \_, 'saint-petersburg') & прямой ход \\
		\hline
	\end{tabular}
\end{table}
\begin{table}
	\begin{tabular}[h!]{|p{1cm}|p{5cm}|p{5cm}|p{5cm}|}
				\hline
	8 & Поиск bank\_client по примеру & bank\_client('mishkina', 'tinkoff', 'payment', 2439027, 8, 'moscow')., не подходит & откат \\
\hline
9 & Поиск bank\_client по примеру & ... & откат \\
\hline

10 & Поиск bank\_client по примеру & bank\_client('toporova', 'tinkoff', 'savings', 60815175, 9, 'saint-petersburg'), Bank = 'tinkoff' & прямой ход \\
\hline
11 & Общий пример найден, вывод & Lname=toporova, Town=saint-petersburg, Street=varshavshkoe sh., Bank=tinkoff, Tel\_num=+79165072034 &  \\
\hline
\end{tabular}
\end{table}

    