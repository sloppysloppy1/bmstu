% ответы на вопросы
\newpage
\section*{Ответы на вопросы}
\addcontentsline{toc}{section}{\tocsecindent{Ответы на вопросы}}
\begin{enumerate}
    \item \textbf{В каком фрагменте программы сформулировано знание?}\\ Правило - это предложение, имееющее следующий вид:
    $A :- B_1,... , B_n$, где 
    A называется заголовком правила, а $B_1,..., B_n$ – телом правила.
    Заголовок содержит отдельное знание о предметной области (составной терм), а тело содержит условия истинности этого знания. Правило называют условной истиной, а факт, не содержащий тела – безусловной истиной.
    
    Заголовок, как составной терм  $f(t_1, t_2, …,t_m)$ , содержит знание о том, что между аргументами: $t_1, t_2, …,t_m$ существует отношение (взаимосвязь, взаимозависимость). А имя этого отношения – это f.  Например,   school(judy, high\_school)      
    
    
    
    \item \textbf{Что содержит тело правила?}\\
    Тело правила содержит условие истинности заголовка правила. 
    
    \item \textbf{Что дает использование переменных при формулировании знаний? В чем отличие формулировки знания с помощью термов с одинаковой арностью при использовании одной переменной и при использовании нескольких переменных? }\\
    Использование переменных в формулировании знаний позволяют уточнять значения и переносить их в пространстве и времени.
    
     Формулировка знаний с использованием переменных носит более общий характер по отношению к знанию, состоящему только лишь из констант. Например, использование знаний с одинаковой арностью при использовании одной переменной носит менее общий характер по отношению знания с использованием нескольких переменных. 

    \item \textbf{С каким квантором переменные входят в правило, в каких пределах переменная уникальна?}\\
    Переменные входят в правило с квантором всеобщности (для любой). Именованные переменные уникальны в пределах одного предложения, анонимные уникальны все. 
	
    \item \textbf{Какова семантика (смысл) предложений раздела DOMAINS?  Когда, где и с какой целью используется это описание?}\\
   	Домены дают возможность присваивать различным видам информации, которая в противном случае выглядела бы одинаково, отличные имена. В программе Visual Prolog объекты в отношении (аргументы предиката) принадлежат доменам; это могут быть домены стандартные или специальные, определяемые программистами.
   	
   	Раздел domains служит двум целям. Во-первых, можно определить для доменов осмысленные имена, причем даже в том случае, если внутренне они совпадают с именами уже существующих доменов. Во-вторых, объявления специальных доменов используются для объявления структур данных, которые стандартными доменами не определяются.
   	
   	Иногда целесообразно объявить домен тогда, когда возникает потребность более четкого выделения каких-либо частей раздела predicates. Объявление программистом своих собственных доменов помогает документировать предикаты, которые определяются путем задания в качестве типа аргумента удобного и понятного имени.
   	
    
    \item \textbf{Какова семантика (смысл) предложений раздела PREDICATES? Когда, и где используется это описание? С какой целью?}\\
    Некоторые предикаты уже известны системе, они называются стандартными или встроенными. Если программист определяет в разделе clauses свой собственный предикат, то он должен объявить его в разделе predicates. В противном случае Visual Prolog не будет знать, о чем идет речь. Когда объявляется предикат, Прологу сообщается о том, к каким доменам принадлежат аргументы этого предиката.
    Предикаты определяются фактами и правилами. В разделе predicates  перечисляется каждый предикат с указанием доменов аргументов.
    

    \item \textbf{Унификация каких термов запускается на самом первом шаге работы системы? Каковы назначение и результат использования алгоритма унификации?}\\
    На первом шаге работы происходит унификация вопроса и первого предложения базы знаний. Алгоритм унификации необходим для попытки "увидеть одинаковость" – сопоставимость двух термов, может завершаться успехом или тупиковой ситуацией. Результат унификации – ответ «да» или «нет». 
    
    \item \textbf{В каком случае запускается механизм отката?}\\
    Механизм отката запускается в 2 случаях:
    \begin{itemize}
    	\item Если алгоритм попал в тупиковую ситуацию. 
    	\item Если резольвента не пуста и решение найдено, но в базе знание остались не отмеченные предложения.
    \end{itemize}  
\end{enumerate}

% ответ
%\\