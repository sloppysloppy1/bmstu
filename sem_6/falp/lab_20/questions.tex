% ответы на вопросы
\newpage
\newpage
\section*{Ответы на вопросы}
\addcontentsline{toc}{section}{\tocsecindent{Ответы на вопросы}}
\begin{enumerate}
	\item \textbf{Как организуется хвостовая рекурсия в Prolog?}\\ 
	Для осуществления хвостовой рекурсии рекурсивный вызов определяемого предиката должен быть последней подцелью в теле рекурсивного правила и к моменту рекурсивного вызова не должно остаться точек возврата (непроверенных альтернатив).
	    
	\item \textbf{Какое первое состояние резольвенты?}\\
	Изначально в резольвенте находится вопрос.
	

    \item \textbf{Каким способом можно разделить список на части, какие, требования к частям?}\\
	Cписок можно разбить на начало и остаток. Начало списка – это группа первых элементов, не менее одного. Остаток списка – обязательно список (может быть пустой). Для разделения списка на начало, и остаток используется вертикальная черта (|) за последним элементом начала. Остаток это всегда один (простой или составной) терм. 
	
	\item \textbf{Как выделить за один шаг первые два подряд идущих элемента списка? Как выделить 1-й и 3-й элемент за один шаг?}
	
	Для этого нужно связать их с переменными: {[First, Second|\_]}, тогда First  будет обозначать первый элемент списка, а Second -- второй). \\Соответственно, конструкция {[First, \_, Third|\_]} используется, чтобы выделить 1-й и 3-й элемент за один шаг (First -- первый элемент списка, Third -- третий)

	


    \item \textbf{Как формируется новое состояние резольвенты?}\\
   	На каждом шаге имеется некоторая совокупность целей - утверждений, истинность (выводимость) которых надо доказать. Эта совокупность называется резольвентой - её состояние меняется в процессе доказательства (Для хранения резольвенты система использует стек). 
   	Новая резольвента образуется в два этапа:
   	\begin{itemize}
   		\item в текущей резольвенте выбирается одна из подцелей (по стековому принципу - верхняя) и для неё выполняется редукция - замена подцели на тело найденного (подобранного, если удалось) правила (а как подбирается правило?),
   		\item затем, к полученной конъюнкции целей применяется подстановка, полученная как наибольший общий унификатор цели (выбранной) и заголовка сопоставленного с ней правила.
   	\end{itemize}	
 
\item \textbf{Когда останавливается работа системы? Как это определяется на формальном уровне?}

Работа системы останавливается в двух случаях:

\begin{itemize}
	\item когда встретился символ отсечения (!);
	\item когда резольвента осталась пустой (формально не осталось подходящих фактов и правил).
\end{itemize}
\end{enumerate}

% ответ
%\\