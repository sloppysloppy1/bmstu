% ответы на вопросы
\newpage
\section*{Ответы на вопросы}
\addcontentsline{toc}{section}{\tocsecindent{Ответы на вопросы}}
\begin{enumerate}
    \item \textbf{Что собой представляет программа на Prolog, какова ее структура?}\\
    	Программа на Prolog представляет собой: базу знаний и вопрос. С помощью подбора ответов на запросы он (Prolog, программа) извлекает хранящуюся (известную в программе) информацию. База знаний содержит истинностные знания, используя которые программа выдает ответ на запрос. Одной из особенностей Prolog является то, что при поиске ответов на вопрос, он рассматривает альтернативные варианты и находит все возможные решения (методом проб и ошибок) — множества значений переменных, при которых на поставленный вопрос можно ответить –«да».
    	\\
    	Программа на Prolog состоит из разделов. Каждый раздел начинается со своего заголовка. Структура программы: 
    	\begin{enumerate}
    		\item директивы компилятора — зарезервированные символьные константы;
    		\item CONSTANTS — раздел описания констант;
    		\item DOMAINS — раздел описания доменов;
    		\item DATABASE — раздел описания предикатов внутренней базы данных;
    		\item PREDICATES — раздел описания предикатов;
    		\item CLAUSES — раздел описания предложений базы знаний;
    		\item GOAL — раздел описания внутренней цели (вопроса).
    	\end{enumerate}
    	В программе не обязательно должны быть все разделы.
     \item \textbf{Как реализуется программа на Prolog?}\\
  		Программа на Prolog не является последовательностью действий, - она представляет собой набор фактов и правил, которые формируют базу знаний о предметной области. Факты представляют собой составные термы, с помощью которых фиксируется наличие истинностных отношений между объектами предметной области — аргументами терма. Правила являются обобщенной формулировкой условия истинности знания – отношения между объектами предметной области (аргументами терма), которое записано в заголовке правила. Условие истинности этого отношения  является телом правила. Заголовок правила отделяется от тела правила символом    «:-»  , правило завершается символом  «.».  <заголовок> :- <тело правила>. \\
  		Заголовок правила — это утверждение базы знаний (предикат), синтаксически это составной терм. Тело правила может представлять собой один терм или быть последовательностью термов (конъюнкцией или дизъюнкцией). В лабораторных работах будем использовать только конъюнкцию (термы в теле разделяются запятыми).
  		Утверждения программы — это предикаты. Предикаты могут не содержать переменных (основные) или содержать переменные (не основные). В процессе выполнения программы — система пытается найти, используя базу знаний , такие значения переменных, при которых на поставленный вопрос можно дать ответ «Да».
  		
    \item \textbf{Как формируются результаты работы программы?}\\
    База знаний обрабатывается сверху - вниз, термы слева - направо.
    \\
    Правила вывода - это утверждения о взаимосвязи между допущениями и заключениями которые справедливы всегда с позиции исчисления предикатов.
    \begin{enumerate}
    	\item Если факты в программе не содержат переменные и вопрос не содержит переменные то по правилу совпадения.
    	\item Если факты содержат переменные а вопрос основной (то есть нет), то применяется правило обобщения фактов.
    	\item Факты и вопросы содержат переменные, для факта выполняется процесс конкретизации а для результата выполняется правило обобщения. 
    \end{enumerate}

\end{enumerate}

% ответ
%\\