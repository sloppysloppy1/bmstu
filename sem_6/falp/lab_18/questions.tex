% ответы на вопросы
\newpage
\section*{Ответы на вопросы}
\addcontentsline{toc}{section}{\tocsecindent{Ответы на вопросы}}
\begin{enumerate}
	\item \textbf{Что такое рекурсия? Как организуется хвостовая рекурсия в Prolog? Как организовать выход из рекурсии в Prolog?}\\
	Рекурсия – это один из способов организации повторных вычислений. \\  
	Для осуществления хвостовой рекурсии рекурсивный вызов определяемого предиката должен быть последней подцелью в теле рекурсивного правила и к моменту рекурсивного вызова не должно остаться точек возврата (непроверенных альтернатив). \\
	Параметры должны изменяться на каждом шаге так, чтобы в итоге либо сработал базис рекурсии, либо условие выхода из рекурсии, размещенное в самом правиле.
	    
	\item \textbf{Какое первое состояние резольвенты?}\\
	Изначально в резольвенте находится вопрос.
	
    \item \textbf{В каком случае система запускает алгоритм унификации? Каково назначение использования алгоритма унификации? Каков  результат работы алгоритма унификации?}\\ 
    Система запускает унификацию в том случае, если ей был задан вопрос. Унификация вопроса и первого предложения базы знаний происходит на первом шаге работы программы. \\
    Алгоритм унификации необходим для попытки "увидеть одинаковость" – сопоставимость двух термов, может завершаться успехом или тупиковой ситуацией. 
    \\
    Результат алгоритма унификации – ответ «да» или «нет». 

    \item \textbf{В каких пределах программы уникальны переменные? }\\
	Именованные переменные уникальны в рамках одного предложения. Анонимная переменная уникальна всегда. Переменные предназначены для передачи значений «во времени и в пространстве». 
	
	\item \textbf{\textbf{Как применяется подстановка, полученная с помощью алгоритма унификации?}}\\
	
	Пока стек не пуст – \textbf{цикл}:
	\begin{itemize}
		\item	считать из стека в рабочую область очередное равенство S=Т
		\item	обработать считанное по правилам:
		\begin{itemize}
			\item	если S и Т несовпадающие константы,			то неудача=1, и выход из цикла
			\item	если одинаковые константы					то следующий шаг цикла
			\item	если S переменная и Т терм содержащий S,		то неудача=1, и выход из цикла
			\item	если S переменная и Т терм НЕ содержащий S,	то отыскать в стеке и в результирующей ячейке все вхождения S и заменить на Т. Добавить в результирующую ячейку равенство S=Т.  Следующий шаг цикла
			\item	если S и Т составные термы с разными функторами или разными арностями, то неудача=1, выход из цикла
			\item	если S и Т составные термы с одинаковыми функторами и арностью: $S=f(s_1,s_2 ...,s_m);$   $T=f(t_1,t_2 ...,t_m),$  то занести в стек равенство $S_1=T_1, S_2=T_2 ... S_m=T_m.$
		\end{itemize}
		\item	очистить рабочее поле
	\end{itemize}
	–  \textbf{конец цикла}

    \item \textbf{Как меняется резольвента?}\\
   	На каждом шаге имеется некоторая совокупность целей - утверждений, истинность (выводимость) которых надо доказать. Эта совокупность называется резольвентой - её состояние меняется в процессе доказательства (Для хранения резольвенты система использует стек). 
   	Новая резольвента образуется в два этапа:
   	\begin{itemize}
   		\item в текущей резольвенте выбирается одна из подцелей (по стековому принципу - верхняя) и для неё выполняется редукция - замена подцели на тело найденного (подобранного, если удалось) правила (а как подбирается правило?),
   		\item затем, к полученной конъюнкции целей применяется подстановка, полученная как наибольший общий унификатор цели (выбранной) и заголовка сопоставленного с ней правила.
   	\end{itemize}	

    \item \textbf{В каком случае запускается механизм отката?}\\
    Механизм отката запускается в 2 случаях:
    \begin{itemize}
    	\item Если алгоритм попал в тупиковую ситуацию. 
    	\item Если резольвента не пуста и решение найдено, но в базе знание остались не отмеченные предложения.
    \end{itemize}  
\end{enumerate}

% ответ
%\\