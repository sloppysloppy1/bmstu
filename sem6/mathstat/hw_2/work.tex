\newpage
\section*{Задание 1}
\large

\addcontentsline{toc}{section}{\tocsecindent{Задание 1}}
\subsection*{Постановка задачи}
Для контроля качества работы молокоперерабатывающего завода были проверены $n=5$ пачек молока номинальной жирностью 3,2\%, в результате чего были получены значения $(\overline{x_n })=3$\%,$S(\overline{x_n })=0.1$\% жирности. Считая распределение контролируемого признака нормальным, при уровне значимости $\alpha=0.01$ с использованием одностороннего критерия проверить гипотезу о том, что средняя жирность молока удовлетворяет заявленному показателю.
\subsection*{Решение}
Введем основную и конкурирующую ей гипотезы:
\begin{eqnarray}H_0 = \{\text{жирность больше или равна 3.2\%}\}\\ H_1 = \{\text{жирность менее 3.2\%}\}\end{eqnarray}
Рассчитаем критерий:
\begin{equation}T_{\text{набл.}}=\frac{(\overline{x_n}-a_0)\sqrt{n}}{S(\overline{x_n })}=\frac{(3-3.2) \cdot \sqrt{5}}{0.1}=-4.47\end{equation}
Критическое значение для односторонней области при $\alpha=0.01$ и $k=n-1=4$ равно 3.75, то есть
\begin{equation}
t_{\text{правост.}}=3.75.
\end{equation}
Таким образом, получается, что
\begin{eqnarray}
t_{\text{левост.}}=-3.75\\T_{\text{набл.}} < t_{\text{левост.}}.
\end{eqnarray}
Это означает, что нулевая гипотеза отвергается, то есть средняя жирность молока при уровне значимости $\alpha=0.01$ не удовлетворяет заявленному показателю.\\
\textbf{Ответ:} гипотеза отвергается.
