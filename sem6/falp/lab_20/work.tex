% --- ugly internals for language definition ---
%
\makeatletter
\lstset{
	captionpos = below,
	frame      = single,
	columns    = fullflexible,
	basicstyle = \small\sffamily,
}
% initialisation of user macros
\newcommand\PrologPredicateStyle{}
\newcommand\PrologVarStyle{}
\newcommand\PrologAnonymVarStyle{}
\newcommand\PrologAtomStyle{}
\newcommand\PrologOtherStyle{}
\newcommand\PrologCommentStyle{}

% useful switches (to keep track of context)
\newif\ifpredicate@prolog@
\newif\ifwithinparens@prolog@

% save definition of underscore for test
\lst@SaveOutputDef{`_}\underscore@prolog

% local variables
\newcount\currentchar@prolog

\newcommand\@testChar@prolog%
{%
	% if we're in processing mode...
	\ifnum\lst@mode=\lst@Pmode%
	\detectTypeAndHighlight@prolog%
	\else
	% ... or within parentheses
	\ifwithinparens@prolog@%
	\detectTypeAndHighlight@prolog%
	\fi
	\fi
	% Some housekeeping...
	\global\predicate@prolog@false%
}

% helper macros
\newcommand\detectTypeAndHighlight@prolog
{%
	% First, assume that we have an atom.
	\def\lst@thestyle{\PrologAtomStyle}%
	% Test whether we have a predicate and modify the style accordingly.
	\ifpredicate@prolog@%
	\def\lst@thestyle{\PrologPredicateStyle}%
	\else
	% Test whether we have a predicate and modify the style accordingly.
	\expandafter\splitfirstchar@prolog\expandafter{\the\lst@token}%
	% Check whether the identifier starts by an underscore.
	\expandafter\ifx\@testChar@prolog\underscore@prolog%
	% Check whether the identifier is '_' (anonymous variable)
	\ifnum\lst@length=1%
	\let\lst@thestyle\PrologAnonymVarStyle%
	\else
	\let\lst@thestyle\PrologVarStyle%
	\fi
	\else
	% Check whether the identifier starts by a capital letter.
	\currentchar@prolog=65
	\loop
	\expandafter\ifnum\expandafter`\@testChar@prolog=\currentchar@prolog%
	\let\lst@thestyle\PrologVarStyle%
	\let\iterate\relax
	\fi
	\advance \currentchar@prolog by 1
	\unless\ifnum\currentchar@prolog>90
	\repeat
	\fi
	\fi
}
\newcommand\splitfirstchar@prolog{}
\def\splitfirstchar@prolog#1{\@splitfirstchar@prolog#1\relax}
\newcommand\@splitfirstchar@prolog{}
\def\@splitfirstchar@prolog#1#2\relax{\def\@testChar@prolog{#1}}

% helper macro for () delimiters
\def\beginlstdelim#1#2%
{%
	\def\endlstdelim{\PrologOtherStyle #2\egroup}%
	{\PrologOtherStyle #1}%
	\global\predicate@prolog@false%
	\withinparens@prolog@true%
	\bgroup\aftergroup\endlstdelim%
}

% language name
\newcommand\lang@prolog{Prolog-pretty}
% ``normalised'' language name
\expandafter\lst@NormedDef\expandafter\normlang@prolog%
\expandafter{\lang@prolog}

% language definition
\expandafter\expandafter\expandafter\lstdefinelanguage\expandafter%
{\lang@prolog}
{%
	language            = Prolog,
	keywords            = {},      % reset all preset keywords
	showstringspaces    = false,
	alsoletter          = (,
	alsoother           = @$,
	moredelim           = **[is][\beginlstdelim{(}{)}]{(}{)},
	MoreSelectCharTable =
	\lst@DefSaveDef{`(}\opparen@prolog{\global\predicate@prolog@true\opparen@prolog},
}

% Hooking into listings to test each ``identifier''
\newcommand\@ddedToOutput@prolog\relax
\lst@AddToHook{Output}{\@ddedToOutput@prolog}

\lst@AddToHook{PreInit}
{%
	\ifx\lst@language\normlang@prolog%
	\let\@ddedToOutput@prolog\@testChar@prolog%
	\fi
}

\lst@AddToHook{DeInit}{\renewcommand\@ddedToOutput@prolog{}}

\makeatother
%
% --- end of ugly internals ---


% --- definition of a custom style similar to that of Pygments ---
% custom colors
\definecolor{PrologPredicate}{RGB}{000,031,255}
\definecolor{PrologVar}      {RGB}{024,021,125}
\definecolor{PrologAnonymVar}{RGB}{000,127,000}
\definecolor{PrologAtom}     {RGB}{186,032,032}
\definecolor{PrologComment}  {RGB}{063,128,127}
\definecolor{PrologOther}    {RGB}{000,000,000}

% redefinition of user macros for Prolog style
\renewcommand\PrologPredicateStyle{\color{PrologPredicate}}
\renewcommand\PrologVarStyle{\color{PrologVar}}
\renewcommand\PrologAnonymVarStyle{\color{PrologAnonymVar}}
\renewcommand\PrologAtomStyle{\color{PrologAtom}}
\renewcommand\PrologCommentStyle{\itshape\color{PrologComment}}
\renewcommand\PrologOtherStyle{\color{PrologOther}}

% custom style definition 
\lstdefinestyle{Prolog-pygsty}
{
	language     = Prolog-pretty,
	upquote      = true,
	stringstyle  = \PrologAtomStyle,
	commentstyle = \PrologCommentStyle,
	literate     =
	{:-}{{\PrologOtherStyle :-}}2
	{,}{{\PrologOtherStyle ,}}1
	{.}{{\PrologOtherStyle .}}1
}



\newpage
\section*{Постановка задачи}
\addcontentsline{toc}{section}{\tocsecindent{Постановка задачи}}

\Large Используя хвостовую рекурсию, разработать, комментируя аргументы, эффективную программу, позволяющую:

\begin{enumerate}
	\item Сформировать список из элементов числового списка, больших заданного значения;
	\item Сформировать список из элементов, стоящих на нечетных позициях исходного списка (нумерация от 0);
	\item Удалить заданный элемент из списка (один или все вхождения);
	\item Преобразовать список в множество (можно использовать ранее разработанные процедуры).
\end{enumerate}
Убедиться в правильности результатов.\\
Для одного из вариантов ВОПРОСА и одного из заданий  составить таблицу, отражающую конкретный порядок работы системы: \\
Т.к. резольвента хранится в виде стека, то состояние резольвенты требуется отображать в столбик: вершина – сверху! Новый шаг надо начинать с нового состояния резольвенты! Для каждого запуска алгоритма унификации, требуется указать № выбранного правила и дальнейшие действия – и почему.



\newpage
\section*{Листинг программы}
\addcontentsline{toc}{section}{\tocsecindent{Листинг программы}}
Ниже представлен листинг программы:
\lstinputlisting[
style      = Prolog-pygsty,
caption    = {Код программы},
]{1.pro}

\section*{Описание работы системы}
\addcontentsline{toc}{section}{\tocsecindent{Описание порядка работы системы}}

Ниже представлен алгоритм поиска ответов на вопрос more\_than([1,2,3], 1, Result).\newpage
\begin{table}[h!]
	\caption{Описание работы системы при поиске ответа на вопрос}
	\begin{tabular}{|p{1.2cm}|p{5cm}|p{4cm}|p{6cm}|}
		\hline
		№ шага & Текущая резольвента –  ТР  & ТЦ, выбираемые правила: сравниваемые термы, 
		подстановка
		& Дальнейшие действия с комментариями \\
		\hline
		1 & \textbf{ТР: more\_than([1,2,3], 1, Result)} & ПРI: [] = [1, 2, 3] \newline
		ПРII:	[Head|Tail] = [1, 2, 3] \newline
		ПРIII:	[Head|Tail] = [1, 2, 3] 
		& ПРI: унификация невозможна => возврат к ТЦ, метка переносится ниже \newline ПРII: успех (подобрано знание) => \{Head = [1] , Tail = [2, 3]\},  проверка тела ПРII, неудача => возврат к ТЦ, метка переносится ниже \newline ПРIII: успех (подобрано знание) => \{Head = [1] , Tail = [2, 3]\},  проверка тела ПРII \\
		\hline
		2 & \textbf{ТР: more\_than([2,3], 1, New\_List)} \newline New\_List = New\_List & ПРI: [] = [2, 3] \newline
		ПРII:	[Head|Tail] = [2, 3]   & ПРI: унификация невозможна => возврат к ТЦ, метка переносится ниже \newline ПРII: успех (подобрано знание) => \{Head = [2], Tail = [3]\},  проверка тела ПРII \\
		\hline
		3 & \textbf{ТР: more\_than([3], 1, New\_List)}  \newline New\_List = [2|[New\_List]]\newline New\_List = New\_List & ПРI: [] = [3] \newline
		ПРII:	[Head|Tail] = [3]  & ПРI: унификация невозможна => возврат к ТЦ, метка переносится ниже \newline ПРII: успех (подобрано знание) => \{Head = [3]\},  проверка тела ПРII \\
		\hline
		
		4 & \textbf{ТР: more\_than([], 1, New\_List)} \newline New\_List = [3|[New\_List]] \newline New\_List = [2|[New\_List]]\newline New\_List = New\_List & ПРI: [] = [] & ПРI: успех (подобрано знание) => пустое тело заменяет цель в резольвенте \\
		\hline
		5 & \textbf{ТР: пусто}  \newline New\_List = []  \newline New\_List = [3|[New\_List]] \newline New\_List = [2|[New\_List]]\newline New\_List = New\_List
		 & New\_List = [2 |[3|[|[]]]] = [2, 3] & успех – однократный ответ – «Да», метка – на ПРI.
		 Отказ от найденного значения (откат), возврат к предыдущему состоянию резольвенты \\
		\hline
		6 & \textbf{ТР: New\_List([])}  & ПРII: [Head|Head] = [] ПРIII: [Head|Tail] = []  & поиск знания от метки ниже \newline унификация невозможна =>неудача, надо включить откат, но 
		метка (метки) в конце процедуры – других альтернатив нет => система завершает работу с единственным результатом – «Да».
		 \\
		\hline
		7 & \textbf{Вывод:} &Result = [2, 3] &  \\
		\hline
	\end{tabular}
\end{table}


