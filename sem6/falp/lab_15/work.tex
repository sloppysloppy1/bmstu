% --- ugly internals for language definition ---
%
\makeatletter
\lstset{
	captionpos = below,
	frame      = single,
	columns    = fullflexible,
	basicstyle = \small\sffamily,
}
% initialisation of user macros
\newcommand\PrologPredicateStyle{}
\newcommand\PrologVarStyle{}
\newcommand\PrologAnonymVarStyle{}
\newcommand\PrologAtomStyle{}
\newcommand\PrologOtherStyle{}
\newcommand\PrologCommentStyle{}

% useful switches (to keep track of context)
\newif\ifpredicate@prolog@
\newif\ifwithinparens@prolog@

% save definition of underscore for test
\lst@SaveOutputDef{`_}\underscore@prolog

% local variables
\newcount\currentchar@prolog

\newcommand\@testChar@prolog%
{%
	% if we're in processing mode...
	\ifnum\lst@mode=\lst@Pmode%
	\detectTypeAndHighlight@prolog%
	\else
	% ... or within parentheses
	\ifwithinparens@prolog@%
	\detectTypeAndHighlight@prolog%
	\fi
	\fi
	% Some housekeeping...
	\global\predicate@prolog@false%
}

% helper macros
\newcommand\detectTypeAndHighlight@prolog
{%
	% First, assume that we have an atom.
	\def\lst@thestyle{\PrologAtomStyle}%
	% Test whether we have a predicate and modify the style accordingly.
	\ifpredicate@prolog@%
	\def\lst@thestyle{\PrologPredicateStyle}%
	\else
	% Test whether we have a predicate and modify the style accordingly.
	\expandafter\splitfirstchar@prolog\expandafter{\the\lst@token}%
	% Check whether the identifier starts by an underscore.
	\expandafter\ifx\@testChar@prolog\underscore@prolog%
	% Check whether the identifier is '_' (anonymous variable)
	\ifnum\lst@length=1%
	\let\lst@thestyle\PrologAnonymVarStyle%
	\else
	\let\lst@thestyle\PrologVarStyle%
	\fi
	\else
	% Check whether the identifier starts by a capital letter.
	\currentchar@prolog=65
	\loop
	\expandafter\ifnum\expandafter`\@testChar@prolog=\currentchar@prolog%
	\let\lst@thestyle\PrologVarStyle%
	\let\iterate\relax
	\fi
	\advance \currentchar@prolog by 1
	\unless\ifnum\currentchar@prolog>90
	\repeat
	\fi
	\fi
}
\newcommand\splitfirstchar@prolog{}
\def\splitfirstchar@prolog#1{\@splitfirstchar@prolog#1\relax}
\newcommand\@splitfirstchar@prolog{}
\def\@splitfirstchar@prolog#1#2\relax{\def\@testChar@prolog{#1}}

% helper macro for () delimiters
\def\beginlstdelim#1#2%
{%
	\def\endlstdelim{\PrologOtherStyle #2\egroup}%
	{\PrologOtherStyle #1}%
	\global\predicate@prolog@false%
	\withinparens@prolog@true%
	\bgroup\aftergroup\endlstdelim%
}

% language name
\newcommand\lang@prolog{Prolog-pretty}
% ``normalised'' language name
\expandafter\lst@NormedDef\expandafter\normlang@prolog%
\expandafter{\lang@prolog}

% language definition
\expandafter\expandafter\expandafter\lstdefinelanguage\expandafter%
{\lang@prolog}
{%
	language            = Prolog,
	keywords            = {},      % reset all preset keywords
	showstringspaces    = false,
	alsoletter          = (,
	alsoother           = @$,
	moredelim           = **[is][\beginlstdelim{(}{)}]{(}{)},
	MoreSelectCharTable =
	\lst@DefSaveDef{`(}\opparen@prolog{\global\predicate@prolog@true\opparen@prolog},
}

% Hooking into listings to test each ``identifier''
\newcommand\@ddedToOutput@prolog\relax
\lst@AddToHook{Output}{\@ddedToOutput@prolog}

\lst@AddToHook{PreInit}
{%
	\ifx\lst@language\normlang@prolog%
	\let\@ddedToOutput@prolog\@testChar@prolog%
	\fi
}

\lst@AddToHook{DeInit}{\renewcommand\@ddedToOutput@prolog{}}

\makeatother
%
% --- end of ugly internals ---


% --- definition of a custom style similar to that of Pygments ---
% custom colors
\definecolor{PrologPredicate}{RGB}{000,031,255}
\definecolor{PrologVar}      {RGB}{024,021,125}
\definecolor{PrologAnonymVar}{RGB}{000,127,000}
\definecolor{PrologAtom}     {RGB}{186,032,032}
\definecolor{PrologComment}  {RGB}{063,128,127}
\definecolor{PrologOther}    {RGB}{000,000,000}

% redefinition of user macros for Prolog style
\renewcommand\PrologPredicateStyle{\color{PrologPredicate}}
\renewcommand\PrologVarStyle{\color{PrologVar}}
\renewcommand\PrologAnonymVarStyle{\color{PrologAnonymVar}}
\renewcommand\PrologAtomStyle{\color{PrologAtom}}
\renewcommand\PrologCommentStyle{\itshape\color{PrologComment}}
\renewcommand\PrologOtherStyle{\color{PrologOther}}

% custom style definition 
\lstdefinestyle{Prolog-pygsty}
{
	language     = Prolog-pretty,
	upquote      = true,
	stringstyle  = \PrologAtomStyle,
	commentstyle = \PrologCommentStyle,
	literate     =
	{:-}{{\PrologOtherStyle :-}}2
	{,}{{\PrologOtherStyle ,}}1
	{.}{{\PrologOtherStyle .}}1
}



\newpage
\section*{Постановка задачи}
\addcontentsline{toc}{section}{\tocsecindent{Постановка задачи}}

\Large Создать базу знаний \textbf{«Собственники»}, дополнив базу знаний, хранящую знания  (лаб. 13):
\begin{enumerate}

	\item «Телефонный справочник»: Фамилия, №тел, Адрес – структура (Город, Улица, №дома, №кв);
	\item «Автомобили»: Фамилия владельца, Марка, Цвет, Стоимость, и др;
	\item « Вкладчики банков»: Фамилия, Банк, счет, сумма, др.
\end{enumerate}
знаниями о дополнительной собственности владельца. Преобразовать знания об автомобиле к форме знаний о собственности.\\
\textbf{Вид собственности (кроме автомобиля)}:
\begin{enumerate}
	\item	Строение, стоимость и другие его характеристики;
	\item	Участок, стоимость и другие его характеристики;
	\item	Водный\_транспорт, стоимость и другие его характеристики.
\end{enumerate}
Описать  и использовать вариантный домен: Собственность. Владелец может иметь, но только один объект каждого вида собственности (это касается и автомобиля), или не иметь некоторых видов собственности. \\
Используя конъюнктивное правило и 
разные формы задания одного вопроса (пояснять для какого №задания – какой вопрос), 
\textbf{обеспечить возможность поиска:}\begin{enumerate}
	\item	Названий всех объектов собственности заданного субъекта,
	\item	Названий и стоимости всех объектов собственности заданного субъекта,
	\item	* Разработать правило, позволяющее найти суммарную стоимость всех объектов собственности заданного субъекта.\end{enumerate}
Для 2-го пунктa и одной фамилии составить таблицу, отражающую конкретный порядок работы системы, с объяснениями порядка работы и особенностей использования доменов (указать конкретные Т1 и Т2 и полную подстановку на каждом шаге).\\
При желании, можно усложнить свою базу знаний, введя варианты: строение: (Дом, офис, торговый центр), участок: (садовый, территория под застройку, территория под агро-работы), Водный\_транспорт: варианты названий.

\newpage
\section*{Листинг программы}
\addcontentsline{toc}{section}{\tocsecindent{Листинг программы}}
Ниже представлен листинг программы:
\lstinputlisting[
style      = Prolog-pygsty,
caption    = {Задания 1 и 2},
]{src/1.pro}

\section*{Описание работы системы}
\addcontentsline{toc}{section}{\tocsecindent{Описание порядка поиска ответов}}

Ниже представлен алгоритм поиска ответов на вопрос get\_property(Name, Property, Cost), Name = klyuge. Всего ответов должно получиться 3.
\begin{table}
	\caption{Описание работы системы на примере вывода названия собственности и ее цены}
\begin{tabular}{|p{1cm}|p{5cm}|p{5cm}|p{5cm}|}
	\hline
	№ шага & Сравниваемые термы; подстановка, если есть  & Результат  & Дальнейшие действия: прямой ход или откат (к чему приводит?) \\
	\hline
	1 & Подстановка (связывание) Name = klyuge в запрос get\_property(Name, Property, Cost)  & get\_property(klyuge, Property, Cost) & прямой ход \\
	\hline
	2 & Поиск совпадений и подстановка (связывание) klyuge, Property и Cost в соответствующее правило  get\_property(X, Y, Z) :- person(X,\_, \_, building (Y,Z,\_))& get\_property(klyuge, Property, Cost) & прямой ход \\
	\hline
	3 & Подстановка X = klyuge, Y = Property и Z = Cost в person(X, \_,\_, building (Y,Z,\_)) & Создается пример person(klyuge, \_, \_, building (Property,Cost,\_)) & прямой ход \\
	\hline
	4 & Поиск person по примерy & person(klyuge, '+7912571026', ad('volgograd', 'varshavshkoe sh.', 52, 83), building(barn, 3213124, ad('volgograd', 'varshavshkoe sh.', 11, 1))), Property = building, Cost = 3213124 & прямой ход \\
	\hline
	5 & Общий пример найден, вывод & Name = klyuge, Property = building, Cost = 3213124 & прямой ход \\
	\hline
	6 & Поиск person по примерy & person(toporova, '+79165072034', ad('saint-petersburg', 'varshavshkoe sh.', 11, 158), building(shed, 123124, ad('saint-petersburg', 'varshavshkoe sh.', 11, 17))), не подходит & прямой ход \\
	\hline
	7 & Поиск person по примеру & ... & прямой ход \\
	\hline
	8 & Поиск person по примеру & person(toporova, '+79165072034', ad('saint-petersburg', 'varshavshkoe sh.', 33, 45), car('mitsubishi', "green", 12836628, 2018, 'saint-petersburg')), не подходит & откат, т.к дошли до конца БЗ (следовательно, тупик)  \\
	\hline
		9 & Поиск совпадений и подстановка (связывание) klyuge, Property и Cost в соответствующее правило  get\_property(X, Y, Z) :- person(X,\_, \_, place (Y,Z,\_,\_))& get\_property(klyuge, Property, Cost) & прямой ход \\
	\hline
\end{tabular}
\end{table}

\begin{table}
\begin{tabular}{|p{1cm}|p{5cm}|p{5cm}|p{5cm}|}
		\hline

	10 & Подстановка X = klyuge, Y = Property и Z = Cost в person(X, \_,\_, place (Y,Z,\_,\_)) & Создается пример person(klyuge, \_, \_, building (Property,Cost,\_,\_)) & прямой ход \\
	\hline
	11 & Поиск person по примерy & person(klyuge, '+7912571026', ad('volgograd', 'varshavshkoe sh.', 52, 83), building(barn, 3213124, ad('volgograd', 'varshavshkoe sh.', 11, 1))), не подходит & прямой ход \\
	\hline
	12 & Поиск person по примеру & ... & прямой ход \\
	\hline
	13 & Поиск person по примеру & person(klyuge, "+7912571026", ad('volgograd', 'varshavshkoe sh.', 52, 83), place(appartment, 12312312, ad('volgograd', 'varshavshkoe sh.', 52, 83), '11.11.1999')), Property = appartment, Cost = 12312312 & прямой ход \\
	\hline
	14 & Общий пример найден, вывод & Name = klyuge, Property = appartment, Cost = 12312312 & прямой ход \\
	\hline
	15 & Поиск person по примерy & person(zayceva, "+79138767153", ad('kasimov', 'lermontovskiy pr.', 83, 578), car('geely', 'white', 6409914, 2015, 'kasimov')), не подходит & прямой ход \\
	\hline
	16 & Поиск person по примеру & ... & прямой ход \\
	\hline
	17 & Поиск person по примеру & person(toporova, '+79165072034", ad('saint-petersburg', 'varshavshkoe sh.', 33, 45), car('mitsubishi', 'green', 12836628, 2018, 'saint-petersburg')), не подходит & откат, т.к дошли до конца БЗ (следовательно, тупик)  \\
		\hline
				18 & Поиск совпадений и подстановка (связывание) klyuge, Property и Cost в соответствующее правило  get\_property(X, Y, Z) :- person(X,\_, \_, person(X, \_, \_, car(Y,\_,Z,\_,\_))& get\_property(klyuge, Property, Cost) & прямой ход \\
		\hline
			19 & Подстановка X = klyuge, Y = Property и Z = Cost в person(X, \_,\_, car(Y,\_,Z,\_,\_)) & Создается пример person(klyuge, \_, \_, car(Property,\_,Cost,\_,\_)) & прямой ход \\
		\hline
\end{tabular}
\end{table}

\begin{table}
	\begin{tabular}{|p{1cm}|p{5cm}|p{5cm}|p{5cm}|}
		\hline

	
		20 & Поиск person по примерy & person(klyuge, '+7912571026', ad('volgograd', 'varshavshkoe sh.', 52, 83), building(barn, 3213124, ad('volgograd', 'varshavshkoe sh.', 11, 1))), не подходит & прямой ход \\
		\hline
		21 & Поиск person по примеру & ... & прямой ход \\
		\hline
		22 & Поиск person по примеру & person(toporova, '+79165072034', ad('saint-petersburg', 'varshavshkoe sh.', 33, 45), car('mitsubishi', 'green', 12836628, 2018, 'saint-petersburg')), не подходит & откат, т.к дошли до конца БЗ (следовательно, тупик) \\
			\hline
		23 & Поиск совпадений и подстановка (связывание) klyuge, Property и Cost в соответствующее правило  get\_property(X, Y, Z) :- person(X,\_, \_, transport(Y,Z,\_)) & get\_property(klyuge, Property, Cost) & прямой ход \\
		\hline
		24 & Подстановка X = klyuge, Y = Property и Z = Cost в person(X, \_,\_, transport (Y,Z,\_)) & Создается пример person(klyuge, \_, \_, building (Property,Cost,\_)) & прямой ход \\
		\hline
		25 & Поиск person по примерy & person(klyuge, '+7912571026', ad('volgograd', 'varshavshkoe sh.', 52, 83), building(barn, 3213124, ad('volgograd', 'varshavshkoe sh.', 11, 1))), не подходит & прямой ход \\
		\hline
		26 & Поиск person по примеру & ... & прямой ход \\
		\hline
		27 & Поиск person по примеру & person(klyuge, '+7912571026", ad('volgograd', 'varshavshkoe sh.', 52, 83), transport('water\_bike', 213124, '21.11.2007'))., Property = 'water\_bike', Cost = 213124 & прямой ход \\
		\hline
		28 & Общий пример найден, вывод & Name = klyuge, Property = 'water\_bike', Cost = 213124 & прямой ход \\
		\hline
		29 & Поиск person по примеру & person(toporova, '+79165072034', ad('saint-petersburg', 'varshavshkoe sh.', 33, 45), car('mitsubishi', 'green', 12836628, 2018, 'saint-petersburg')), не подходит & откат, т.к дошли до конца БЗ (следовательно, тупик)  \\
		\hline
			Вывод: & подстановка & Т.к. стек пуст - успех и рез. ячейке подстановка
			 &  \\
		\hline
	\end{tabular}
\end{table}