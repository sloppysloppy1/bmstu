% --- ugly internals for language definition ---
%
\makeatletter
\lstset{
	captionpos = below,
	frame      = single,
	columns    = fullflexible,
	basicstyle = \small\sffamily,
}
% initialisation of user macros
\newcommand\PrologPredicateStyle{}
\newcommand\PrologVarStyle{}
\newcommand\PrologAnonymVarStyle{}
\newcommand\PrologAtomStyle{}
\newcommand\PrologOtherStyle{}
\newcommand\PrologCommentStyle{}

% useful switches (to keep track of context)
\newif\ifpredicate@prolog@
\newif\ifwithinparens@prolog@

% save definition of underscore for test
\lst@SaveOutputDef{`_}\underscore@prolog

% local variables
\newcount\currentchar@prolog

\newcommand\@testChar@prolog%
{%
	% if we're in processing mode...
	\ifnum\lst@mode=\lst@Pmode%
	\detectTypeAndHighlight@prolog%
	\else
	% ... or within parentheses
	\ifwithinparens@prolog@%
	\detectTypeAndHighlight@prolog%
	\fi
	\fi
	% Some housekeeping...
	\global\predicate@prolog@false%
}

% helper macros
\newcommand\detectTypeAndHighlight@prolog
{%
	% First, assume that we have an atom.
	\def\lst@thestyle{\PrologAtomStyle}%
	% Test whether we have a predicate and modify the style accordingly.
	\ifpredicate@prolog@%
	\def\lst@thestyle{\PrologPredicateStyle}%
	\else
	% Test whether we have a predicate and modify the style accordingly.
	\expandafter\splitfirstchar@prolog\expandafter{\the\lst@token}%
	% Check whether the identifier starts by an underscore.
	\expandafter\ifx\@testChar@prolog\underscore@prolog%
	% Check whether the identifier is '_' (anonymous variable)
	\ifnum\lst@length=1%
	\let\lst@thestyle\PrologAnonymVarStyle%
	\else
	\let\lst@thestyle\PrologVarStyle%
	\fi
	\else
	% Check whether the identifier starts by a capital letter.
	\currentchar@prolog=65
	\loop
	\expandafter\ifnum\expandafter`\@testChar@prolog=\currentchar@prolog%
	\let\lst@thestyle\PrologVarStyle%
	\let\iterate\relax
	\fi
	\advance \currentchar@prolog by 1
	\unless\ifnum\currentchar@prolog>90
	\repeat
	\fi
	\fi
}
\newcommand\splitfirstchar@prolog{}
\def\splitfirstchar@prolog#1{\@splitfirstchar@prolog#1\relax}
\newcommand\@splitfirstchar@prolog{}
\def\@splitfirstchar@prolog#1#2\relax{\def\@testChar@prolog{#1}}

% helper macro for () delimiters
\def\beginlstdelim#1#2%
{%
	\def\endlstdelim{\PrologOtherStyle #2\egroup}%
	{\PrologOtherStyle #1}%
	\global\predicate@prolog@false%
	\withinparens@prolog@true%
	\bgroup\aftergroup\endlstdelim%
}

% language name
\newcommand\lang@prolog{Prolog-pretty}
% ``normalised'' language name
\expandafter\lst@NormedDef\expandafter\normlang@prolog%
\expandafter{\lang@prolog}

% language definition
\expandafter\expandafter\expandafter\lstdefinelanguage\expandafter%
{\lang@prolog}
{%
	language            = Prolog,
	keywords            = {},      % reset all preset keywords
	showstringspaces    = false,
	alsoletter          = (,
	alsoother           = @$,
	moredelim           = **[is][\beginlstdelim{(}{)}]{(}{)},
	MoreSelectCharTable =
	\lst@DefSaveDef{`(}\opparen@prolog{\global\predicate@prolog@true\opparen@prolog},
}

% Hooking into listings to test each ``identifier''
\newcommand\@ddedToOutput@prolog\relax
\lst@AddToHook{Output}{\@ddedToOutput@prolog}

\lst@AddToHook{PreInit}
{%
	\ifx\lst@language\normlang@prolog%
	\let\@ddedToOutput@prolog\@testChar@prolog%
	\fi
}

\lst@AddToHook{DeInit}{\renewcommand\@ddedToOutput@prolog{}}

\makeatother
%
% --- end of ugly internals ---


% --- definition of a custom style similar to that of Pygments ---
% custom colors
\definecolor{PrologPredicate}{RGB}{000,031,255}
\definecolor{PrologVar}      {RGB}{024,021,125}
\definecolor{PrologAnonymVar}{RGB}{000,127,000}
\definecolor{PrologAtom}     {RGB}{186,032,032}
\definecolor{PrologComment}  {RGB}{063,128,127}
\definecolor{PrologOther}    {RGB}{000,000,000}

% redefinition of user macros for Prolog style
\renewcommand\PrologPredicateStyle{\color{PrologPredicate}}
\renewcommand\PrologVarStyle{\color{PrologVar}}
\renewcommand\PrologAnonymVarStyle{\color{PrologAnonymVar}}
\renewcommand\PrologAtomStyle{\color{PrologAtom}}
\renewcommand\PrologCommentStyle{\itshape\color{PrologComment}}
\renewcommand\PrologOtherStyle{\color{PrologOther}}

% custom style definition 
\lstdefinestyle{Prolog-pygsty}
{
	language     = Prolog-pretty,
	upquote      = true,
	stringstyle  = \PrologAtomStyle,
	commentstyle = \PrologCommentStyle,
	literate     =
	{:-}{{\PrologOtherStyle :-}}2
	{,}{{\PrologOtherStyle ,}}1
	{.}{{\PrologOtherStyle .}}1
}



\newpage
\section*{Постановка задачи}
\addcontentsline{toc}{section}{\tocsecindent{Постановка задачи}}

\Large Используя хвостовую рекурсию, разработать программу, позволяющую найти 
\begin{enumerate}
	\item \textbf{n!};
	\item \textbf{n-е число Фибоначчи}.
\end{enumerate}

Убедиться в правильности результатов.
Для каждого случая пункта 2 обосновать необходимость всех условий тела. 
Для одного из вариантов ВОПРОСА и каждого варианта задания 2 составить таблицу, отражающую конкретный порядок работы системы: \\
Т.к. резольвента хранится в виде стека, то состояние резольвенты требуется отображать в столбик: вершина – сверху! Новый шаг надо начинать с нового состояния резольвенты!


\newpage
\section*{Листинг программы}
\addcontentsline{toc}{section}{\tocsecindent{Листинг программы}}
Ниже представлен листинг программы:
\lstinputlisting[
style      = Prolog-pygsty,
caption    = {Задания 1 и 2},
]{src/1.pro}

\section*{Описание работы системы}
\addcontentsline{toc}{section}{\tocsecindent{Описание порядка поиска ответов}}

Ниже представлен алгоритм поиска ответов на вопросы fact(2, Result) и fibb(2, Result).

\begin{table}
	\caption{Описание работы системы при решении факториала}
	\begin{tabular}{|p{1.2cm}|p{5cm}|p{5cm}|p{5cm}|}
		\hline
		№ шага & Состояние резольвенты, и вывод: дальнейшие действия (почему?) & Для каких термов запускается алгоритм унификации: Т1=Т2 и каков результат (и подстановка) & Дальнейшие действия: прямой ход или откат (почему и к чему приводит?) \\
		\hline
		1 & Резольвента: \newline fact(2, Result). Начинается поиск совпадений по БЗ &  & прямой ход \\
		\hline
		2 & Резольвента: \newline fact(2, Result). & Нашли подходящее правило: fact(Curr, Next) :-  Cur = Curr - 1, fact(Cur, Res), Next = Res * Curr. Подставляем Curr = 2, Next = Result & прямой ход \\
		\hline
		3 & Резольвента: \newline Cur = Curr - 1, \newline fact(2, Result). & Пробуем cвязать: Сur = 2 - 1 = 1, удача, идем дальше.   & прямой ход, подставляем Cur = 1 в fact(Cur, Res)\\
		\hline
		4 & Резольвента: \newline fact(1, Res), \newline fact(2, Result). & Поиск совпадений по БЗ. Нашли подходящее правило: fact(Curr, Next) :-  Cur = Curr - 1, fact(Cur, Res), Next = Res * Curr. Подставляем Curr = 1, Next = Res  & прямой ход \\
		\hline
		5 & Резольвента: \newline Cur = Curr - 1, \newline fact(2, Result). & Пробуем cвязать: Сur = 1 - 1 = 0, удача, идем дальше.   & прямой ход, подставляем Cur = 0 в fact(Cur, Res) \\
		\hline
		6 & Резольвента: \newline fact(0, Res), \newline fact(1, Res), \newline fact(2, Result). & Поиск совпадений по БЗ. Нашли подходящее правило: fact(0, 1) :- !. Подставляем Res = 1, один из вопросов решен.  & прямой ход \\
		\hline
		7 & Резольвента: \newline fact(1, Res), \newline fact(2, Result). & Пробуем cвязать: Next = Res * Curr = 1 * 1, (Res = 1 из предыдущего шага). Подставляем Res = 1, один из вопросов решен.  & прямой ход \\
		\hline
		8 & Резольвента: \newline fact(2, Result). & Пробуем cвязать: Next = Res * Curr = 1 * 2, (Res = 1 из предыдущего шага). Подставляем Result = Next = 2, вопрос решен.  & прямой ход \\
		\hline
		\textbf{Вывод}& Result = 2 &  & \\
		\hline
	\end{tabular}
\end{table}

\begin{table}
	\caption{Описание работы системы при решении факториала}
	\begin{tabular}{|p{1.2cm}|p{5cm}|p{5cm}|p{5cm}|}
		\hline
		№ шага & Состояние резольвенты, и вывод: дальнейшие действия (почему?) & Для каких термов запускается алгоритм унификации: Т1=Т2 и каков результат (и подстановка) & Дальнейшие действия: прямой ход или откат (почему и к чему приводит?) \\
		\hline
		1 & Резольвента: \newline fibb(2, Result). Начинается поиск совпадений по БЗ &  & прямой ход \\
		\hline
		2 & Резольвента: \newline fibb(2, Result). & Нашли подходящее правило: fibb(Curr, Next) :-  Cur = Curr - 1, Prev = Cur - 1, fibb(Cur, Cur\_Res), fibb(Prev, Prev\_Res), Next = Cur\_Res + Prev\_Res. Подставляем Curr = 2, Nex\t = Result & прямой ход \\
		\hline
		3 & Резольвента: \newline Cur = Curr - 1, \newline fact(2, Result). & Пробуем cвязать: Сur = 2 - 1 = 1, удача, идем дальше.   & прямой ход, подставляем Cur = 1 в fact(Cur, Res)\\
		\hline
		4 & Резольвента: \newline Prev = Cur - 1, \newline fact(2, Result). & Пробуем cвязать: Prev = 1 - 1 = 0, удача, идем дальше.   & прямой ход, подставляем Cur = 1 в fibb(Cur, Cur\_Res)\\
		\hline
		5 & Резольвента: \newline fibb(1, Cur\_Res) \newline fibb(2, Result). & Поиск совпадений по БЗ. Нашли подходящее правило: fibb(1, 1) :- !. Подставляем Cur\_Res = 1, один из вопросов решен.  & прямой ход, подставляем Prev = 0 в fact(Prev, Prev\_Res) \\
		\hline
		6 & Резольвента: \newline fibb(0, Prev\_Res) \newline fibb(2, Result). & Поиск совпадений по БЗ. Нашли подходящее правило: fibb(0, 0) :- !. Подставляем Prev\_Res = 0, один из вопросов решен.  & прямой ход\\
		\hline
		8 & Резольвента: \newline fibb(2, Result). & Пробуем cвязать: Next = Cur\_Res + Prev\_Res = 1 + 0. Подставляем Result = Next = 1, вопрос решен.  & прямой ход \\
		\hline
		\textbf{Вывод}& Result = 1 &  & \\
		\hline
	\end{tabular}
\end{table}

