% --- ugly internals for language definition ---
%
\makeatletter
\lstset{
	captionpos = below,
	frame      = single,
	columns    = fullflexible,
	basicstyle = \small\sffamily,
}
% initialisation of user macros
\newcommand\PrologPredicateStyle{}
\newcommand\PrologVarStyle{}
\newcommand\PrologAnonymVarStyle{}
\newcommand\PrologAtomStyle{}
\newcommand\PrologOtherStyle{}
\newcommand\PrologCommentStyle{}

% useful switches (to keep track of context)
\newif\ifpredicate@prolog@
\newif\ifwithinparens@prolog@

% save definition of underscore for test
\lst@SaveOutputDef{`_}\underscore@prolog

% local variables
\newcount\currentchar@prolog

\newcommand\@testChar@prolog%
{%
	% if we're in processing mode...
	\ifnum\lst@mode=\lst@Pmode%
	\detectTypeAndHighlight@prolog%
	\else
	% ... or within parentheses
	\ifwithinparens@prolog@%
	\detectTypeAndHighlight@prolog%
	\fi
	\fi
	% Some housekeeping...
	\global\predicate@prolog@false%
}

% helper macros
\newcommand\detectTypeAndHighlight@prolog
{%
	% First, assume that we have an atom.
	\def\lst@thestyle{\PrologAtomStyle}%
	% Test whether we have a predicate and modify the style accordingly.
	\ifpredicate@prolog@%
	\def\lst@thestyle{\PrologPredicateStyle}%
	\else
	% Test whether we have a predicate and modify the style accordingly.
	\expandafter\splitfirstchar@prolog\expandafter{\the\lst@token}%
	% Check whether the identifier starts by an underscore.
	\expandafter\ifx\@testChar@prolog\underscore@prolog%
	% Check whether the identifier is '_' (anonymous variable)
	\ifnum\lst@length=1%
	\let\lst@thestyle\PrologAnonymVarStyle%
	\else
	\let\lst@thestyle\PrologVarStyle%
	\fi
	\else
	% Check whether the identifier starts by a capital letter.
	\currentchar@prolog=65
	\loop
	\expandafter\ifnum\expandafter`\@testChar@prolog=\currentchar@prolog%
	\let\lst@thestyle\PrologVarStyle%
	\let\iterate\relax
	\fi
	\advance \currentchar@prolog by 1
	\unless\ifnum\currentchar@prolog>90
	\repeat
	\fi
	\fi
}
\newcommand\splitfirstchar@prolog{}
\def\splitfirstchar@prolog#1{\@splitfirstchar@prolog#1\relax}
\newcommand\@splitfirstchar@prolog{}
\def\@splitfirstchar@prolog#1#2\relax{\def\@testChar@prolog{#1}}

% helper macro for () delimiters
\def\beginlstdelim#1#2%
{%
	\def\endlstdelim{\PrologOtherStyle #2\egroup}%
	{\PrologOtherStyle #1}%
	\global\predicate@prolog@false%
	\withinparens@prolog@true%
	\bgroup\aftergroup\endlstdelim%
}

% language name
\newcommand\lang@prolog{Prolog-pretty}
% ``normalised'' language name
\expandafter\lst@NormedDef\expandafter\normlang@prolog%
\expandafter{\lang@prolog}

% language definition
\expandafter\expandafter\expandafter\lstdefinelanguage\expandafter%
{\lang@prolog}
{%
	language            = Prolog,
	keywords            = {},      % reset all preset keywords
	showstringspaces    = false,
	alsoletter          = (,
	alsoother           = @$,
	moredelim           = **[is][\beginlstdelim{(}{)}]{(}{)},
	MoreSelectCharTable =
	\lst@DefSaveDef{`(}\opparen@prolog{\global\predicate@prolog@true\opparen@prolog},
}

% Hooking into listings to test each ``identifier''
\newcommand\@ddedToOutput@prolog\relax
\lst@AddToHook{Output}{\@ddedToOutput@prolog}

\lst@AddToHook{PreInit}
{%
	\ifx\lst@language\normlang@prolog%
	\let\@ddedToOutput@prolog\@testChar@prolog%
	\fi
}

\lst@AddToHook{DeInit}{\renewcommand\@ddedToOutput@prolog{}}

\makeatother
%
% --- end of ugly internals ---


% --- definition of a custom style similar to that of Pygments ---
% custom colors
\definecolor{PrologPredicate}{RGB}{000,031,255}
\definecolor{PrologVar}      {RGB}{024,021,125}
\definecolor{PrologAnonymVar}{RGB}{000,127,000}
\definecolor{PrologAtom}     {RGB}{186,032,032}
\definecolor{PrologComment}  {RGB}{063,128,127}
\definecolor{PrologOther}    {RGB}{000,000,000}

% redefinition of user macros for Prolog style
\renewcommand\PrologPredicateStyle{\color{PrologPredicate}}
\renewcommand\PrologVarStyle{\color{PrologVar}}
\renewcommand\PrologAnonymVarStyle{\color{PrologAnonymVar}}
\renewcommand\PrologAtomStyle{\color{PrologAtom}}
\renewcommand\PrologCommentStyle{\itshape\color{PrologComment}}
\renewcommand\PrologOtherStyle{\color{PrologOther}}

% custom style definition 
\lstdefinestyle{Prolog-pygsty}
{
	language     = Prolog-pretty,
	upquote      = true,
	stringstyle  = \PrologAtomStyle,
	commentstyle = \PrologCommentStyle,
	literate     =
	{:-}{{\PrologOtherStyle :-}}2
	{,}{{\PrologOtherStyle ,}}1
	{.}{{\PrologOtherStyle .}}1
}



\newpage
\section*{Постановка задачи}
\addcontentsline{toc}{section}{\tocsecindent{Постановка задачи}}

\Large \textbf{Создать базу знаний: «ПРЕДКИ»}, позволяющую наиболее эффективным способом (за меньшее количество шагов, что обеспечивается меньшим количеством предложений БЗ - правил), используя разные варианты (примеры) одного вопроса, \textbf{определить} (указать: какой вопрос для какого варианта):
\begin{enumerate}
	\item по имени субъекта определить всех его бабушек (предки 2-го колена),
	\item по имени субъекта определить всех его дедушек (предки 2-го колена),
	\item по имени субъекта определить всех его бабушек и дедушек (предки 2-го колена),
	\item по имени субъекта определить его бабушку по материнской линии (предки 2-го колена),
	\item по имени субъекта определить его бабушку и дедушку по материнской линии (предки 2-го колена).
\end{enumerate}
	
Минимизировать количество правил и количество вариантов вопрпосов. Использовать конъюнктивные правила и простой вопрос.
Для одного из вариантов \textbf{ВОПРОСА и конкретной БЗ} составить таблицу, отражающую конкретный порядок работы системы, с объяснениями: 
очередная проблема на каждом шаге и метод ее решения; 
каково новое текущее состояние резольвенты, как получено;
какие дальнейшие действия? (Запускается ли алгоритм унификации? Каких термов? Почему этих?); вывод по результатам очередного шага и дальнейшие действия.

Т.к. резольвента хранится в виде стека, то состояние резольвенты требуется отображать в столбик: вершина – сверху! Новый шаг надо начинать с нового состояния резольвенты!


\newpage
\section*{Листинг программы}
\addcontentsline{toc}{section}{\tocsecindent{Листинг программы}}
Ниже представлен листинг программы:
\lstinputlisting[
style      = Prolog-pygsty,
caption    = {Задания 1 и 2},
]{src/1.pro}

\section*{Описание работы системы}
\addcontentsline{toc}{section}{\tocsecindent{Описание порядка поиска ответов}}

Ниже представлен алгоритм поиска ответов на вопрос grandparents(oleg, w, Grandmother\_name, \_). Всего должен получиться один ответ.

\begin{table}
	\caption{Описание работы системы на примере поиска бабушек}
	\begin{tabular}{|p{1cm}|p{5cm}|p{5cm}|p{5cm}|}
		\hline
		№ шага & Состояние резольвенты, и вывод: дальнейшие действия (почему?) & Для каких термов запускается алгоритм унификации: Т1=Т2 и каков результат (и подстановка) & Дальнейшие действия: прямой ход или откат (почему и к чему приводит?) \\
		\hline
		1 & grandparents(oleg, w, Grandmother\_name, \_) &  & прямой ход \\
		\hline
		1 & grandparents(oleg, w, Grandmother\_name, \_) & Находим подходящее правило grandparents(X, Y, Z, B) и связываем (подставляем) переменные:  X=oleg, Y=w, Z= Grandmother\_name & прямой ход \\
		\hline
		2 & mom\_or\_dad(Z, Y, A) \newline grandparents(oleg, w, Grandmother\_name, \_) & Подставляем Z= Grandmother\_name, Y=w в mom\_or\_dad(Z, Y, A). Получаем пример mom\_or\_dad(Grandmother\_name, w, A) & прямой ход \\
		\hline
		2 & mom\_or\_dad(Z, Y, A) \newline grandparents(oleg, w, Grandmother\_name, \_) & Поиск совпадений по БЗ, находим mom\_or\_dad(anya, w, oleg). Z = anya, A = oleg & прямой ход \\
		\hline
		3 & mom\_or\_dad(A, B, X) \newline grandparents(oleg, w, Grandmother\_name, \_) & Подставляем A = oleg, X=oleg в mom\_or\_dad(A, B, X). Получаем пример mom\_or\_dad(oleg, B, oleg) & прямой ход \\
		\hline
		3 & mom\_or\_dad(A, B, X) \newline grandparents(oleg, w, Grandmother\_name, \_) & Поиск совпадений по БЗ, ничего не находим & откат, снова решаем предыдущий вопрос mom\_or\_dad(Z, Y, A) \\
		\hline
		4 & mom\_or\_dad(Z, Y, A) \newline grandparents(oleg, w, Grandmother\_name, \_) & Поиск совпадений по БЗ, находим mom\_or\_dad(masha, w, anya). Z = masha, A = anya & прямой ход \\
		\hline
		4 & mom\_or\_dad(Z, Y, A) \newline grandparents(oleg, w, Grandmother\_name, \_) & Подставляем A = anya, X=oleg в mom\_or\_dad(A, B, X). Получаем пример mom\_or\_dad(anya, B,oleg) & прямой ход \\
		\hline
		5 & mom\_or\_dad(A, B, X) \newline grandparents(oleg, w, Grandmother\_name, \_) & Поиск совпадений по БЗ, находим mom\_or\_dad(anya, w, oleg). B = w & прямой ход \\
		\hline
		6 & grandparents(oleg, w, Grandmother\_name, \_) & Подстановка Grandmother\_name=masha, ответ получен. & откат, поскольку не дошли до конца Б3 \\
		\hline
		7 & mom\_or\_dad(Z, Y, A) \newline grandparents(oleg, w, Grandmother\_name, \_) & Поиск совпадений по БЗ, ничего не находимим & прямой ход \\
		\hline
		Вывод: & Grandmother\_name=masha & & конец  \\
		\hline
	\end{tabular}
\end{table}
